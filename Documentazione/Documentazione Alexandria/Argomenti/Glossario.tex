
\newglossaryentry{Alexandria}
{
    name={Alexandria},
    description={Alexandria è un applicativo capace di gestire e creare i riferimenti bibliografici}
}

\newglossaryentry{riferimento}
{
    name={riferimento},
    description={Un riferimento è un collegamento ad un'opera bibliografica esistente dotato di nome, data, descrizione, codici univoci e di un eventuale URL per la visualizzazione}
}

\newglossaryentry{autore}
{
    name={autore},
    description={Un autore è una persona che ha prodotto un'opera da cui è poi tratto un riferimento presente nell'applicazione}
}

\newglossaryentry{categoria}
{
    name={categoria},
    description={Una categoria è un insieme contenente più riferimenti che trattano la stessa collezione di argomenti}
}

\newglossaryentry{sopra-categoria}
{
    name={sopra-categoria},
    description={Una sopra-categoria è un insieme contenente una o più categorie. La relazione che accomuna le categorie dell'insieme può essere varia e a scelta dell'utente. Infine, una sopra-categoria non può contenere se stessa o una categoria che ha già tale sopra-categoria come sopra-categoria}
}

\newglossaryentry{attributo}
{
    name={attributo},
    description={Un attributo è un dettaglio di un riferimento e può essere della seguente natura: uno o più codici univoci, il titolo del riferimento, la data di pubblicazione, una descrizione del riferimento, il nome dell'autore, il nome della casa editrice, il numero di edizione ed eventuale titolo a cui si riferisce}
}

\newglossaryentry{casi d'uso}
{
    name={casi d'uso},
    description={Un caso d'uso è una delle funzionalità principali del sistema. Senza di esse, l'applicativo non sarebbe completo e permette di svolgere una delle attività principali richieste}
}


\printglossaries

