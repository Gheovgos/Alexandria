
\newglossaryentry{Alexandria}
{
    name={Alexandria},
    description={Alexandria è un applicativo capace di gestire e creare i riferimenti bibliografici}
}

\newglossaryentry{riferimento}
{
    name={riferimento},
    description={Un riferimento è un collegamento ad un'opera bibliografica esistente dotato di nome, data, descrizione, codici univoci e di un eventuale URL per la visualizzazione}
}

\newglossaryentry{autore}
{
    name={autore},
    description={Un autore è una persona che ha prodotto un'opera da cui è poi tratto un riferimento presente nell'applicazione}
}

\newglossaryentry{categoria}
{
    name={categoria},
    description={Una categoria è un insieme contenente più riferimenti che trattano la stessa collezione di argomenti}
}

\newglossaryentry{sopra-categoria}
{
    name={sopra-categoria},
    description={Una sopra-categoria è un insieme contenente una o più categorie. La relazione che accomuna le categorie dell'insieme può essere varia e a scelta dell'utente. Infine, una sopra-categoria non può contenere se stessa o una categoria che ha già tale sopra-categoria come sopra-categoria}
}

\newglossaryentry{attributo}
{
    name={attributo},
    description={Un attributo è un dettaglio di un riferimento e può essere della seguente natura: uno o più codici univoci, il titolo del riferimento, la data di pubblicazione, una descrizione del riferimento, il nome dell'autore, il nome della casa editrice, il numero di edizione ed eventuale titolo a cui si riferisce}
}

\newglossaryentry{casi d'uso}
{
    name={casi d'uso},
    description={Un caso d'uso è una delle funzionalità principali del sistema. Senza di esse, l'applicativo non sarebbe completo e permette di svolgere una delle attività principali richieste}
}

\newglossaryentry{tipi di riferimento}
{
    name={tipo di rifeirmento},
    description={Un tipo di riferimento è il tipo di formato del riferimento pubblicato e può assumere diversi tipi, ovvero: un libro, un articolo, un fascicolo, una rivista, una conferenza. La conferenza può essere di formato visivo virtuale}
}

\newglossaryentry{tipo di ricerca}
{
    name={tipo di ricerca},
    description={Il tipo di ricerca determina in \textit{quale modo} debba essere ricercato un determinato riferimento e può essere per titolo (mostra solo i riferiementi avente tale titolo), per autore (mostra tutti i riferimenti di quel determinato autore) e per DOI (mostra i riferimenti avente quel determinato DOI)}
}

\newglossaryentry{DOI}
{
    name={DOI},
    description={Un DOI è un codice univoco digitale di un particolare riferimento.}
}

\newglossaryentry{layer}
{
    name={Layer},
    description={Un layer è uno strato dell'architettura del sistema. Può essere gerarchico e svolge una determinata attività o compito.}
}

\newglossaryentry{HTTP}
{
    name={HTTP},
    description={HTTP è un protocollo per la comunicazione delle informazioni fra client-server. In Alexandria, essa è utilizzata per la comunicazione fra l'applicativo e il DBMS}
}

\newglossaryentry{REST}
{
    name={REST},
    description={Representational state transfer è uno stile architetturale per sistemi distribuiti. In questo caso, è la rappresentazione dell'Architettura di Alexandria.}
}

\newglossaryentry{DBMS}
{
    name={DBMS},
    description={Un DBMS (Database management system) è un sistema di gestione di una base di dati, in particolare per la creazione, manipolazione e ricerca dei dati. In Alexandria, si riferisce a PostgreSQL, un DBMS di tipo relazionale.}
}


\newglossaryentry{front end}
{
    name={Front-End},
    description={Il front end è l'insieme di tutte le attività e operazioni che l'utente finale deve compiere o vedere.}
}

\newglossaryentry{API}
{
    name={API},
    description={Un'API è un'interfaccia di programmazione di un'applicazione, ovvero un insieme di librerie e framework atte a implementare determinate funzionalità o un sottosistema. In Alexandria, abbiamo implementato un'API apposita per la ricerca efficiente dei riferimenti.}
}

\newglossaryentry{Spring Boot}
{
    name={Spring Boot},
    description={Spring Boot è un'API rest per l'implementazione, appunto, di un applicativo basato su REST.}
}

\printglossaries

